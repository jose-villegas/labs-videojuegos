\begin{center}
\textsc{\Large Laboratorio 15}~\\
{\large Vídeo Juegos, Programación, Diseño}~\\
\emph{Networking, Conexión a Internet y Multijugador}
\end{center}

\section{Pre-Laboratorio}
\todo[inline]{Por hacer.}

\section{Introducción}
Los vídeo juegos multijugador son aquellos donde dos a mas personas interaccionan en el mismo juego, la forma en la que interaccionan depende el objetivo y alcance del vídeo juego, esta puede ser de forma cooperativa o competitiva.

Existen distintas formas de permitir la adición de múltiples jugadores una es de forma local utilizada usualmente en vídeo juegos para computadores personales donde varios jugadores se conectan a través de una red local, otra forma es de forma remota a través del Internet.

La programación del networking en un vídeo juego consiste en administrar la data enviada y recibida a través de diferentes métodos de conexión de tal forma que el software responda apropiadamente a distintas acciones entre varios jugadores \cite{valve_networking}.

\section{Actividad}
\todo[inline]{Por hacer.}