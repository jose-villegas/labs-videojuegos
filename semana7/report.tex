\begin{center}
\textsc{\Large Laboratorio 7}~\\
\emph{\large Vídeo Juegos, Física en los Vídeo Juegos}
\end{center}

\section{Pre-Laboratorio}
\todo[inline]{Por hacer.}

\section{Introducción}
Para agregar realismo, nuevas mecánicas o mayor calidad visual se introducen leyes físicas dentro del motor de juego, es mayormente usado en juegos tridimensionales. Estas nuevos efectos se introducen en forma de simulaciones las cuales son aproximaciones de fenómenos reales utilizando valores discretos.

\section{Simulaciones Físicas}
Hay dos clases centrales de simulaciones fisicas, simulaciones de cuerpos rígidos (\emph{rigid-body physics}) y simulaciones de cuerpos blandos (\emph{soft-body physics}). En una simulación de cuerpos rígidos los objetos se agrupan entre categorías basadas en como deberían interactivo, las simulaciones de cuerpos rígidos son menos intensas en cuanto a perdida de \emph{performance}. Las simulaciones de cuerpos blandos consisten en simular secciones individuales de cada objeto de tal forma que este se comporte de manera realista, usualmente utilizadas para simular objetos deformables como ropa o materiales destructibles.

\section{Sistemas de Partículas}


\section{Físicas Ragdoll}

\section{Proyectiles}

\section{Actividad}
\todo[inline]{Por hacer.}