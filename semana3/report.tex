\begin{center}
\textsc{\Large Laboratorio 3}~\\
{\large Vídeo Juegos, Programación}~\\
\emph{Interacción y \emph{Feedback} con Dispositivos de Entrada}
\end{center}

\section{Pre-Laboratorio}
\begin{itemize}
\item Investigar los siguientes conceptos:
\begin{enumerate}
  \item Feedback.
  \item Háptica o Haptics.
\end{enumerate}
\item Investigue al menos 3 dispositivos de entrada utilizados en los vídeo juegos.
\end{itemize}

\section{Introducción}
\setlength\intextsep{0pt}
\begin{wrapfigure}[11]{l}{0.3\linewidth}
\includegraphics[width=\linewidth]{media/wasd.jpg}
\caption{La combinación de teclas \emph{WASD} es comúnmente utilizadas para controlar el movimiento de un personaje.}
\label{fig:wasd}
\end{wrapfigure}
Los vídeo juegos son un medio interactivo que requiere del input del usuario, por esto los juegos cuentan con distintos tipos de dispositivos de entrada que proporcionan \emph{feedback} visual al realizar alguna acción con dicho dispositivo. El dispositivo mas conocido es el control de vídeo juego o mando, utilizado comúnmente en consolas, en otras plataformas el dispositivo principal puede variar como en el caso de los juegos de computador personal el uso del mouse y teclado es mas común, los dispositivos de entrada no suelen ser exclusivos de la plataformas por tanto existen mandos para el computador personal e incluso teclados para las consolas \cite[p.~395]{jenkinscreatinggames}.
\section{Feedback}
La interacción básica entre un jugador y un juego es sencilla, si el jugador hace el algo entonces el juego hace también algo en forma de respuesta. Esto se le llama \emph{feedback} y es lo que diferencia a un juego de cualquier otra medio de entretenimiento como películas o música \cite[p.~18]{bobbatesgamedesign}.

Cada entrada al juego debe tener una respuesta discernible. Esta entrada puede tener muchas formas dependiendo de la plataforma y los dispositivos de entrada, la respuesta usual es de tipo visual, auditiva e incluso táctil en caso de estar utilizando un control de juego con vibración.

\section{Actividad}
Continuando con la programación de la lógica de juego en la actividad del laboratorio dos usted ahora debe seguir programando la base del juego y lógica principal de todos los actores en su mundo de juego.
\begin{itemize}
\item Debe cumplir los siguientes objetivos.
\begin{enumerate}
  \item Agregar variadas acciones relacionadas con la mecánica de juego al jugador principal, estas acciones deben ahora ser realizadas utilizando teclado y mouse.
  \item Programar lógica de actores en el mundo de juego, expandir la lógica de los actores que interaccionan con el jugador para completar su idea en cuanto mecánicas de juego.
  \item Agregar serie de acciones relacionadas con el estado general del juego (pausa, salir, reiniciar), estas deben ser ejecutadas a través del teclado.
\end{enumerate}
\end{itemize}