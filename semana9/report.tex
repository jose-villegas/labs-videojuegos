\begin{center}
\textsc{\Large Laboratorio 9}~\\
{\large Vídeo Juegos, Programación, Diseño}~\\
\emph{Niveles y Diseño de Niveles}
\end{center}

\section{Pre-Laboratorio}
\todo[inline]{Por hacer.}

\section{Introducción}
Un nivel, mapa, área, mundo, zona o fase en un vídeo juego es el espacio total disponible al jugador  mientras este mismo completa algún objetivo concreto de el juego.

\emph{En juegos con una progresión lineal los niveles son áreas que forman parte de un mundo mas grande, los juegos suelen conectar niveles con cada nivel representando una zona o localidad en un mundo de mayor tamaño, esto suele suceder por dos razones, primero diseño del juego y segundo por razones de \emph{performance}.

Cada nivel usualmente tiene asociado algún objetivo diseñado para el mismo, este objetivo puede ser tan sencillo como moverse de un punto a otro, es usual que cuando el objetivo sea completado el jugador sea trasladado a otro nivel o el siguiente nivel en caso de un juego con progresión lineal, es usual también que si el jugador falla completando el objetivo este deba volver a completar el nivel o retornar a algún punto en el mismo.
}\section{Actividad}
\todo[inline]{Por hacer.}