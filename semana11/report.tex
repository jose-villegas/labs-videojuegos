\begin{center}
\textsc{\Large Laboratorio 11}~\\
{\large Vídeo Juegos, Diseño, Programación}~\\
\emph{Arte, Assets, Texturas y Modelos}
\end{center}

\section{Pre-Laboratorio}
\todo[inline]{Por hacer.}

\section{Introducción}
\setlength\intextsep{0pt}
\begin{wrapfigure}[9]{r}{0.4\linewidth}
\includegraphics[width=\linewidth]{semana11/okamips2.jpg}
\caption{\emph{Okami} \cite{okami} es un juego con arte inspirado en una forma de dibujo llamada sumi-e \cite{sumie}.}
\label{fig:particles}
\end{wrapfigure}
Buen arte se ha convertido en una forma de de juzgar los vídeo juegos. Un inmensa mayoría compra y gana interés en algún vídeo juego a partir de como estos se ven, esto es una reacción lógica ya que desde una tienda virtual o física o a través de vídeos o tomas de pantallas no se puede evaluar el gameplay del juego en cuestión \cite[p.~171]{bobbatesgamedesign}..

Los artistas afectan ahora una inmensa parte del diseño del juego, desde el diseño de la interfaz de juego hasta hasta el ambiente general presentado en el universo del juego y los efectos especiales. La creación de arte ha aumentando en complejidad a través de los años y el crecimiento de los vídeo juegos como una industria, de tal forma las herramientas de creación de arte también cada vez son mas complejas.
\section{Actividad}
\todo[inline]{Por hacer.}