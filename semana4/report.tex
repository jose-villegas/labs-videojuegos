\begin{center}
\textsc{\Large Laboratorio 4}~\\
{\large Vídeo Juegos, Programación}~\\
\emph{Interacción con Dispositivos de Entrada}
\end{center}

\section{Pre-Laboratorio}
\todo[inline]{Por hacer.}

\section{Introducción}
\begin{wrapfigure}[11]{l}{0.3\linewidth}
\includegraphics[width=\linewidth]{semana4/wasd.jpg}
\caption{La combinación de teclas \emph{WASD} es comúnmente utilizadas para controlar el movimiento de un personaje.}
\label{fig:wasd}
\end{wrapfigure}
Los vídeo juegos son un medio interactivo que requiere del input del usuario, por esto los juegos cuentan con distintos tipos de dispositivos de entrada que proporcionan \emph{feedback} visual al realizar alguna acción con dicho dispositivo. El dispositivo mas conocido es el control de vídeo juego o mando, utilizado comúnmente en consolas, en otras plataformas el dispositivo principal puede variar como en el caso de los juegos de computador personal el uso del mouse y teclado es mas común, los dispositivos de entrada no suelen ser exclusivos de la plataformas por tanto existen mandos para el computador personal e incluso teclados para las consolas.

\section{Actividad}
\todo[inline]{Por hacer.}