\begin{center}
\textsc{\Large Laboratorio 6}~\\
{\large Vídeo Juegos, Diseño, Programación}~\\
\emph{Audio, Sonido y Música}
\end{center}

\section{Pre-Laboratorio}
\todo[inline]{Por hacer.}

\section{Introducción}
Los primeros vídeo juegos no poseían audio hoy día los vídeo juegos son un medio audio-visual. El audio en los vídeo juegos puede ser separado en tres categorías: efectos de sonido, música y voces \cite{erikgamedevelopment}\cite{valve_audio}.

Los efectos de sonido son producidos modificando \emph{samples} de audio o replicando y grabando dichos efectos de sonido con objetos reales. Los efectos de sonido son importantes para proporcionar inmersión. 

La música puede ser electrónica o sintetizada, o producida con instrumentos en vivo. Existen varios contextos en donde la música es presentada en el juego por ejemplo la música ambiental que busca reforzar el estado de animo o apariencia del juego, música especifica para menús y créditos, música para batallas, persecuciones o cualquier otro evento rápido y tenso.

Las voces son utilizadas para reforzar las interacciones entre personajes y sus personalidades.

\section{Actividad}
\todo[inline]{Por hacer.}