\begin{center}
\textsc{\Large Laboratorio 6}~\\
{\large Vídeo Juegos, Diseño, Programación}~\\
\emph{Audio, Sonido y Música}
\end{center}

\section{Pre-Laboratorio}
\begin{itemize}
\item Investigar lo siguiente:
\begin{enumerate}
  \item Derechos de autor y licencias.
  \item OPL u Open Content License.
\end{enumerate}
\item De algún juego que conozca, analice:
\begin{enumerate}
  \item ¿La musica cambia durante algún menú o interfaz? En caso de hacerlo o no, a que cree que se deba esto.
  \item ¿La música durante el gameplay de alguna forma se siente que 'encaja' con el ambiente del juego?
  \item ¿Qué efectos de sonido son comunes en el juego al usar o interaccionar con objetos? (eje. disparar un arma, caminar o chocar con algo) ¿Cual cree que sea el propósito de estos sonidos?
\end{enumerate}
\end{itemize}


\section{Introducción}
Los primeros vídeo juegos no poseían audio hoy día los vídeo juegos son un medio audio-visual. El audio en los vídeo juegos puede ser separado en tres categorías: efectos de sonido, música y voces \cite{erikgamedevelopment}\cite{valve_audio}.

Los efectos de sonido son producidos modificando \emph{samples} de audio o replicando y grabando dichos efectos de sonido con objetos reales. Los efectos de sonido son importantes para proporcionar inmersión. 

La música puede ser electrónica o sintetizada, o producida con instrumentos en vivo. Existen varios contextos en donde la música es presentada en el juego por ejemplo la música ambiental que busca reforzar el estado de animo o apariencia del juego, música especifica para menús y créditos, música para batallas, persecuciones o cualquier otro evento rápido y tenso \cite[p.~188]{bobbatesgamedesign}.

Las voces son utilizadas para reforzar las interacciones entre personajes y sus personalidades.

\section{Actividad}
\todo[inline]{Por hacer.}