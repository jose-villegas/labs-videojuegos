\begin{center}
\textsc{\Large Laboratorio 10}~\\
{\large Vídeo Juegos, Programación, Diseño}~\\
\emph{Puntuaciones, Objetivos y Motivación}
\end{center}

\section{Pre-Laboratorio}
\todo[inline]{Por hacer.}

\section{Introducción}
La importancia de una experiencia de juego depende de el interés general que este juego pueda generar. Crear y mantener el interés de el jugador es una forma de manejar su grado de motivación. La motivación del jugador es el factor que va determinar si el jugador va continuar jugando después de cierto tiempo e incluso después de terminar el juego \cite[p.~75]{jenkinscreatinggames}.

En un inicio se tiene la ventaja de saber que el jugador esta motivado al comenzar el juego porque esta ya ha tomado los primeros pasos de obtener e iniciar el juego. Luego de esto es que empieza nuestro trabajo como desarrollados y diseñadores de mantener al jugador motivado y entretenido. Es usual que los primeros minutos de juego suelan definir si se capta o no al jugador \cite{motivationdesign}.
\section{Sistemas de Motivación}
El diseño del juego debe construir un ciclo sobre las necesidades del jugador y responder a estas con una sucesión de retos y recompensas. Esta estructura en vídeo juego esta construida alrededor de los principios de progresión, avance y logros \cite{motivationdesign}.

Crear un ciclo reto/recompensa es relativamente sencillo pero puede fácilmente volverse repetitivo y restrictivo. Es tarea de los desarrolladores y diseñadores mantener al jugador motivado en el ciclo reto/recompensa, evitando los problemas principales que pueden convertir a este ciclo en un juego tedioso.
\subsection{Motivación en las Recompensas}
Un sistema de recompensas es base fundamental en para mantener la motivación. Básicamente consiste en que cada esfuerzo del jugador debe tener alguna forma de recompensa, esta recompensa puede tomas muchas formas y su rol principal es por supuesto que motivar al jugador.

En juegos de action-RPG como \emph{Diablo} \cite{diablo} o \emph{Guild Wars} \cite{guildwars} el gameplay gira alrededor del poder que tiene el personaje manejado por el jugador, este poder va creciendo a medida que el jugador progresa en el juego, la recompensa es un personaje mucho mas poderoso que el personaje a medida que se progresa en el juego. Este progreso también abre acceso a nuevas áreas y objetos del juego.
\subsection{Motivación en las Necesidades}
Comúnmente utilizado en juegos de estrategia, construcción o \emph{managing}. En juegos como \emph{StarCraft} \cite{starcraft} la mecánica completa del juego consiste en la adquisición y control de los recursos en un mapa, para progresar en el juego el jugador actualmente necesita los recursos proveídos por el juego.
\subsection{Motivación en el Reto}
Usual en juegos competitivos como juegos de pelea o juegos de deportes, en este caso la motivación del jugador existe gracias al reto que el juego proporciona, la necesidad del jugador es cada vez ser mejor y saber mas del juego, de manera que este se sienta preparado para nuevos retos, la recompensa es la victoria ya que esta prueba el esfuerzo hecho por el jugador.
\section{Sistemas de Puntuación}
Un sistema de puntuación es una buena forma de proveer motivación al jugador.  Es parte integral en un sistema de recompensas, esto permite premiar al jugador y ademas confirmar su éxito en alguna acción o forma de reto en el juego.

El jugador es premiado con puntos y/u otorgándole un rango en el juego. Esta puntuación determina el progreso del jugador, es incluso posible agregar "bonos" a los premios por objetivos extra o premiando la eficiencia del jugador completando alguna tarea. El jugador crea entonces una relación lógica donde el universo de juego esta organizado y estructurado en un sistema de valores, donde a mayor esfuerzo, mayor recompensa, donde el progreso en el juego facilita el esfuerzo y da acceso a mejores recompensas.
\section{Actividad}
\todo[inline]{Por hacer.}